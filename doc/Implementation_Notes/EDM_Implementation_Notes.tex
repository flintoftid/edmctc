\documentclass[a4paper]{article}
%
\usepackage[utf8]{inputenc}
\usepackage[T1]{fontenc}
\usepackage[english]{babel}
\usepackage{amsmath,esint}
\usepackage{amssymb,amsfonts,textcomp}
\usepackage{color}
\usepackage[top=2.54cm,bottom=1.251cm,left=2.54cm,right=2.54cm,nohead,includefoot,foot=1.289cm,footskip=2.4780002cm]{geometry}
\usepackage{array}
\usepackage{supertabular}
\usepackage{hhline}
\usepackage{hyperref}
\hypersetup{colorlinks=true, linkcolor=blue, citecolor=blue, filecolor=blue, urlcolor=blue, pdftitle=Implementation of the electromagnetic diffusion model of canonical test cases using the finite element method, pdfauthor=Ian Flintoft, pdfkeywords=power balance; diffusion; statistical energy balance}
\usepackage[pdftex]{graphicx}
\usepackage[nottoc]{tocbibind}
\usepackage{stmaryrd}
\usepackage{natbib}
\setcitestyle{authoryear,open={(},close={)}}
%\usepackage{times}
\usepackage{fancyhdr}
\usepackage[reqno]{empheq}
%\usepackage{txfonts}
\usepackage{doi}
\usepackage[nodayofweek,level]{datetime}
\newcommand{\mydate}{\formatdate{3}{6}{2017}}
%
\include{MathMacros}
\numberwithin{equation}{section}
\makeatletter
\newcommand\arraybslash{\let\\\@arraycr}
\makeatother
% Footnote rule
\setlength{\skip\footins}{0.119cm}
\renewcommand\footnoterule{\vspace*{-0.018cm}\setlength\leftskip{0pt}\setlength\rightskip{0pt plus 1fil}\noindent\textcolor{black}{\rule{0.25\columnwidth}{0.018cm}}\vspace*{0.101cm}}
\setlength\tabcolsep{1mm}
\renewcommand\arraystretch{1.3}
\newcounter{Table}
\renewcommand\theTable{\arabic{Table}}
\newcounter{Figure}
\renewcommand\theFigure{\arabic{Figure}}
\providecommand\oiint{\oint}
\newcommand*{\defeq}{\stackrel{\text{def}}{=}}
%
\title{Implementation of electromagnetic diffusion model canonical test cases using FreeFEM++}
\author{Ian Flintoft}
\date{2017-02-04}
\renewcommand{\headrulewidth}{0.0pt}
\renewcommand{\footrulewidth}{0.4pt}
\lhead{}
\chead{}
\rhead{}
\lfoot{Ian Flintoft \href{mailto:ian.flintoft@googlemail.com}{<ian.flintoft@googlemail.com>}}
\rfoot{\mydate}
\pagestyle{fancy}
%
\begin{document}
\clearpage\setcounter{page}{1}{\centering
\textbf{\Large Implementation of electromagnetic diffusion model canonical test cases using FreeFEM++}
\par}
\vspace{5mm}
\thispagestyle{plain}
{\centering\large 
Ian Flintoft\footnote{ Email: \href{mailto:ian.flintoft@googlemail.com}{ian.flintoft@googlemail.com}; 
Web: \url{https://idflintoft.bitbucket.io}}, University of York
\par}
\vspace{5mm}
{\centering
\mydate
\par}
\vspace{5mm}
\textbf{\textit{Abstract}}\textit{ -- These informal notes provide details of the implementation of the 
electromagnetic diffusion model of some canonical test cases using the finite element method
with FreeFEM++. The treatment is terse and no attempt has been made to motivate the methods used or to provide
comprehensive background and references. 
}
%
\setcounter{tocdepth}{3}
\renewcommand\contentsname{Contents}
\tableofcontents
%
\newpage

\section*{Abbreviations}
\label{sc:abbrev}

\begin{supertabular}{m{2cm}l}
ACS              & (Average) Absorption Cross-Section  \\
AE               &(Average) Absorption Efficiency      \\
BC               &Boundary Condition                   \\
DOF              &Degrees of Freedom                   \\
EC               &(Energy) Exchange Coefficient        \\
EDM              &Electromagnetic diffusion Model      \\
EEBC             &Energy Exchange Boundary Condition   \\
EM               &Electromagnetic                      \\
FCBC             &Flux Continuity Boundary Condition   \\
FE               &Finite Element                       \\
FEM              &Finite Element Method                \\
MFP              &Mean-Free-Path                       \\
ODE              &Ordinary Differential Equation       \\
PDE              &Partial Differential Equation        \\
PWB              &Power Balance                        \\
TCS              &(Average) Transmission Cross-Section \\
TE               &(Average) Transmission Efficiency    \\
\end{supertabular}

\newpage
\section*{Symbols}
\label{sc:symbols}

\begin{center}
\begin{supertabular}{|c|c|c|l|}
\hline
\textbf{Symbol}     &\textbf{Unit}   &\textbf{Variable}           &\textbf{Definition} \\
\hline
$w\prt$             &J\,m$^{-3}$     &\texttt{w}                  &Average energy density in cavity \\
$D\pr$              &m$^2$\,s$^{-1}$ &\texttt{D}                  &Diffusivity (Diffusion Coefficient) \\
$\Lambda _V$        &s$^{-1}$        &\texttt{volLossrate}        &Volumetric absorption loss rate \\
$p\prt$             &W\,m$^{-3}$     &\texttt{volSrcDensity}      &Volumetric power density source \\
$\rc_0$             &m\,s${-1}$      &\texttt{c0}                 &Speed of light in free-space \\
$\lambda$           &m               &\texttt{MFP}                &Mean free path \\
$\vJ\prt$           &W\,m$^{-2}$     &\texttt{Jx, Jy, Jz}         &Fickian energy density flux \\
$V$                 &m$^3$           &\texttt{cavityVolume}       &Domain volume \\
$S$                 &m$^2$           &\texttt{cavityArea}         &Domain surface area \\
$\kappa$            &-               &\texttt{specularity}        &Empirical specular correction factor for diffusivity \\
$h\pr$              &m\,s$^{-1}$     &\texttt{EC}                 &Energy exchange coefficient of domain boundary \\
$\alpha\pr$         &-               &\texttt{AE}                 &Average absorption efficiency of domain boundary \\
$\uvn$              &-               &-                           &Outward unit normal vector for domain \\
$\lambda_\rw$       &m               &\texttt{wallMFP}            &Mean free path for wall scattering \\
$D_\rw$             &m$^2$\,s$^{-1}$ &\texttt{wallD}              &Diffusivity for wall scattering \\
$\alpha_\rw\pr$     &~               &\texttt{wallAE}             &Average absorption efficiency of walls \\
$h_\rw\pr$          &m\,s$^{-1}$     &\texttt{wallEC}             &Energy exchange coefficient of walls \\
$\lambda _\rf$      &m               &\texttt{fittingMFP}         &Mean free path for fitting scattering \\
$D_\rf$             &m$^2$\,s$^{-1}$ &\texttt{fittingD}           &Diffusivity path for fitting scattering \\
$\alpha_\rf\pr$     &-               &\texttt{fittingAE}          &Average absorption efficiency of fittings \\
$\sigma_\rf^\rs$    &m$^2$           &\texttt{fittingSCS}         &Average scattering cross-section of fittings \\
$h_\rf\pr$          &m\,s$^{-1}$     &\texttt{fittingEC}          &Energy exchange coefficient of fittings \\
$n_V^\rf$           &m$^{-3}$        &\texttt{fittingConc}        &Volumetric number density of fittings \\
$\Lambda_V^\rf$     &s$^{-1}$        &\texttt{fittingVolLossRate} &Volumetric absorption loss rate due to fittings \\
$P^\rt$             &W               &\texttt{TsrcRP}             &Source total (isotropic) radiated power \\
$V_\rs$             &m$^3$           &\texttt{srcVolume}          &Volume source volume \\
$S_\rs$             &m$^2$           &\texttt{srcArea}            &Surface source surface area \\
$J_\rs$             &W\,m$^{-2}$     &\texttt{srcSurfaceExitance} &Surface source exitance \\
$J_\rc$             &W\,m$^{-1}$     &\texttt{srcLineExitance}    &Contour source line exitance \\
$\tau$              &-               &\texttt{TE}                 &Average transmission efficiency of lossless aperture \\
$w_\rr\prt$         &J\,m$^{-3}$     &\texttt{wr}                 &Average reverberant energy density in cavity \\
$\vJ_\rr\prt$       &W\,m$^{-2}$     &\texttt{Jrx, Jyx, Jrz}      &Reverberant energy density flux \\
$S\prt$             &W\,m$^{-2}$     &\texttt{S}                  &Average scalar power density \\
\hline
\end{supertabular}
\end{center}

\newpage

\section[Summary of the model]{Summary of the model}
\label{sc:sum}

\subsection[The diffusion equation]{The diffusion equation}
\label{sc:sum:edm}

In the electromagnetic diffusion model (EDM) the diffuse electromagnetic energy
density, $w\prt$, at position $\vr$ and time $t$ in a closed
domain $\Omega$ is assumed to satisfy the diffusion equation
\begin{align}
\begin{matrix}
\frac{\partial w\prt}{\partial t} - \vnabla\vcdot\left[D\pr \vnabla w\prt \right] + \Lambda_V w\prt = p\prt &\vr\in\Omega 
\end{matrix}
\end{align}
where $D\pr$ is the potentially inhomogeneous diffusivity, $\Lambda_V$ is an 
average volumetric absorption loss rate and  $p\prt$ is the volume power density 
of any sources~\citep{Picaut1999,Valeau2006,Navarro2015,Savioja2015}. This is a second order parabolic partial differential equation (PDE).
For time dependent problems the initial value of the energy density must be
prescribed over the whole domain
\begin{align}
\begin{matrix}
\left.w\prt\right|_{t_0} = w_0\pr &\vr\in\Omega
\end{matrix}
\end{align}
The diffusivity accounts for the scattering within the cavity. To first
approximation in a large sparsely populated domain it may be regarded as constant and 
is related to the mean-free-path (MFP) between scattering events, $\lambda$, by
\begin{align}
D = \frac{1}{3}\rc_0\lambda 
\end{align}
where $\rc_0$ is the speed of light in free-space. 

The transport of average energy density is described by Fick’s law for the
energy density flux
\begin{align}
\vJ\prt = -D\pr\vnabla w\prt .
\end{align}
The diffusion equation can therefore be interpreted as an equation of continuity
\begin{align}
\begin{matrix}
\frac{\partial w\prt}{\partial t} +\vJ\prt + \Lambda_V w\prt= p\prt &\vr{\in}\Omega
\end{matrix}
\end{align}
for a flow of diffuse energy carried by pseudo-particles.

\subsection[Scattering from the cavity walls]{Scattering from the cavity walls}
\label{sc:sum:walls}

For an empty cavity without a strong aspect ratio (i.e. one that is approximately cubic) 
the MFP for scattering from the walls is given by~\citep{Navarro2015}
\begin{align}
\lambda_\rw = \frac{4V}{S},
\end{align}
where $V$ and $S$ are the interior volume and surface area of the cavity
walls:
\begin{align}
V &= \iiint_{\Omega }\dif{V} \\
S &= \oiint_{\partial\Omega }\dif{S} .
\end{align}
For cavities with one dimension much greater than the others the diffusivity becomes inhomogeneous, 
depending on the location of the source and the disposition of lossses in the cavity~\citep{Visentin2012,Visentin2015}.
For long cavities with relatively low loss a better estimate of the mean-free-path may be given by
\begin{align}
\lambda_\rw = \sqrt{\frac{S}{4\pi}} ;
\end{align}
however, this is a gross approximation which may be of limited scope and accuracy.

\subsection[Scattering and absorption from fittings]{Scattering and absorption from fittings}
\label{sc:sum:fit}

If there are objects (fittings) in the cavity the overall MFP is the harmonic
mean of the MFP of the walls and fittings,
\begin{align}
\frac{1}{\lambda} = \frac{1}{\lambda_\rw} + \frac{1}{\lambda_\rf},
\end{align}
as is the corresponding overall diffusivity~\citep{Valeau2007}
\begin{align}
\frac{1}{D} = \frac{1}{D_\rw} + \frac{1}{D_\rf}.
\end{align}
The diffusivity is therefore dominated by the shortest MFP. The MFP for
scattering with identical fittings of surface area $S_\rf$ and volumetric
concentration $n_V^\rf=N_\rf/V$ can be estimated as~\citep{Valeau2007}
\begin{align}
\lambda_\rf = \frac{1}{\sigma_\rf^\rs n_V^\rf} = \frac{4V}{S_\rf N_\rf},
\end{align}
where $\sigma_\rf^s=S_\rf/4$ is the average scattering cross-section of the
fittings. If the fittings are lossy then a term
\begin{align}
\Lambda_V^\rf = \frac{\rc_0\alpha_\rf}{\lambda_\rf} = \rc_0 n_V^\rf \sigma_\rf^\ra
\end{align}
must also be added to the volumetric scattering rate, $\Lambda_V$, that accounts
for volumetric absorption processes. Here $\alpha_\rf$ is the average
absorption efficiency of the fittings and $\sigma_\rf^\ra=\alpha_\rf S_\rf/4$ the
average absorption cross-section.

Attempts have also been made to include the effects of specular reflections by
introducing an empirical factor, ${\kappa}$, into the diffusivity~\citep{Foy2009} 
\begin{align}
D \rightarrow \kappa D .
\end{align}
Empirically it is found that $1\leq\kappa\lesssim 5$ and that $\kappa$ depends on the 
scattering coefficient of the cavity walls, $s$, which is defined as the fraction
of energy reflected diffusely ($0 \leq s \leq 1$). A simple empirical model was proposed
in ~\citep{Foy2009}:
\begin{align}
\kappa(s) = -2.238\log_\re(s) + 1.549 \,\,\,\,\,\,s\in [10^{-3},1],
\end{align}
though it was noted that the reverberation time was is only accurate for $s>0.4$.

\subsection[Absorption in the cavity walls]{Absorption in the cavity walls}
\label{sc:sum:abs}

Absorption in the domain boundary is described by inhomogeneous Robin (also
called Fourier or Type 3) boundary conditions:
\begin{align}
\begin{matrix}
\uvn\vcdot\left[D\pr\vnabla w\prt\right] + h\pr w\prt + J_\mathrm{in}\prt - J_\mathrm{out}\prt = 0 &\vr\in\partial\Omega
\end{matrix}
\end{align}
where $\uvn$ is an outward normal vector to the domain and 
$h\pr$ is an energy exchange coefficient (EC). $J_\mathrm{in}\prt$ and
$J_\mathrm{out}\prt$ represent an in-flux and out-flux (exitance) of energy density to/from the surface 
respectively and can be used to model sources and energy exchange boundary conditions (EEBCs) as noted below.
Various models have been used in the acoustics literature to relate the exchange coefficient to the
average absorption efficiency of the domain boundary, $\alpha\pr$~\citep{Jing2008,Xiang2009}. The
simplest is based on Sabine's estimate of the reverberation time~\citep{Sabine1922}
\begin{align}
\label{eq:ecsabine}
\begin{matrix}
h\pr = \frac{1}{4}\rc_0\alpha \left(r\right)&0\leq\alpha\lesssim 0.2
\end{matrix}
\end{align}
and is regarded in acoustics as only accurate for low absorption $\alpha\lesssim 0.2$. 
A modified form
\begin{align}
\begin{matrix}
h\pr = -\rc_0\frac{\log_{10}\left(1-\alpha\pr\right)}{4} &0\leq\alpha< 1
\end{matrix}
\end{align}
based on the Eyring reverberation time has also been applied~\citep{Eyring1930}.
This empirically provides better results for moderate absorption, $\alpha\lesssim 0.5$, 
but becomes singular for $\alpha=1$. More recently, a relation based on
a radiative transport model has been proposed which appears to give good
results over the full range of absorption efficiency~\citep{Jing2008}:
\begin{align}
\label{eq:ecjing}
\begin{matrix}
h\pr =\rc_0\frac{\alpha\pr}{2\left(2-\alpha\pr\right)} & 0\leq\alpha\leq 1 .
\end{matrix}
\end{align}

The absorption efficiencies and exchange coefficients for the walls are denoted 
by $\alpha_\rw\pr$ and $h_\rw\pr$ respectively. Large fittings are more
accurately modelled by including their surfaces in the domain boundary; in 
this case their absorption efficiencies and exchange coefficients are denoted by
$\alpha_\rf\pr$ and $h_\rf\pr$ respectively.

\subsection[Sources]{Sources}
\label{sc:sum:src}

A point source of instantaneous total radiated power (TRP) $P^\rt(t)$
located at $\vr_\rs$ can be implemented as a power density
\begin{align}
p\prt = P^\rt(t) \delta^{(3)}\left(\vr-\vr_\rs\right).
\end{align}
The energy density near such a source contains a spurious direct term that can
be subtract to give the reverberant energy density~\citep{Visentin2012}; in 3D this correction is:
\begin{align}
w_\rr\pr = w\pr - \frac{P^\rt}{4\pi D\left|\vr-\vr_\rs\right|}.
\end{align}
The true direct energy density is given by the inverse-square-law for the
isotropic spreading of the power from the source:
\begin{align}
w_\rd\pr = \frac{P^\rt}{4\pi\rc_0\left|\vr-\vr_\rs\right|^2}.
\end{align}
The corresponding reverberant energy density flux is
\begin{align}
\vJ_\rr\prt = -D\vnabla w_\rr\pr = \vJ\prt - \frac{P^\rt}{4\pi\left|\vr-\vr_\rs\right|^3}\left(\vr-\vr_\rs\right).
\end{align}

A uniform volumetric source over a sub-volume of the domain $\Omega_s$ can be implemented
using
\begin{align}
\begin{matrix} 
p\prt = \frac{P^\rt(t)}{V_s} &\vr\in\Omega_\rs 
\end{matrix} \\
V_s = \iiint_{\Omega_\rs}\dif{V}
\end{align}
The spurious direct energy density from such source is less than for point
sources.

A portion of the domain boundary $\partial\Omega_\rs$ can also be used as a
surface source by introducing a source term into the Robin BC:
\begin{align}
\begin{matrix}
\uvn\vcdot\left[D\pr\vnabla w\prt\right] - J_\rs\prt = 0 &\vr\in\partial\Omega_\rs.
\end{matrix}
\end{align}
Here the exitance, $J_\rs\prt$, is
\begin{align}
J_\rs\prt &= \frac{P^\rt(t)}{S_\rs} \\
S_\rs &= \oiint_{\partial\Omega_\rs}\dif{S} .
\end{align}

\subsection[Coupled domains]{Coupled domains}
\label{sc:sum:coup}

Consider the case of two cavities forming domains $\Omega_1$ and $\Omega_2$ with
diffusivities $D_1\pr$ and $D_2\pr$ coupled by a translucent part of their 
shared boundary; the domain 1 side of the shared boundary is denoted by 
${\partial}\Omega_{12}$ and the domain 2 side by ${\partial}\Omega_{21}$.
If the coupling between the cavities is not too large we assume each cavity 
satisfies a diffusion equation with diffusivity given by the standard single 
cavity relationships above
\begin{align}
\begin{matrix}
\frac{{\partial}w_1\prt}{{\partial}t}-{\vnabla}{\vcdot}\left[D_1\pr{\vnabla}w_1\prt\right]+\Lambda_{V;1}w_1\prt = p_1\prt &\vr\in\Omega_1
\end{matrix} \\
\begin{matrix}
\frac{{\partial}w_2\prt}{{\partial}t}-{\vnabla}{\vcdot}\left[D_2\pr{\vnabla}w_2\prt\right]+\Lambda_{V;2}w_2\prt = p_2\prt &\vr\in\Omega_2
\end{matrix}
\end{align}
and that on the non-shared parts of the walls normal Robin boundary conditions
apply
\begin{align}
\begin{matrix}
\uvn_1{\vcdot}\left[D_1\pr{\vnabla}w_1\prt\right]+h_1\pr w_1\prt = 0 &\vr\in\partial\Omega_1/\partial\Omega_{12}
\end{matrix}\\
\begin{matrix}
\uvn_2{\vcdot}\left[D_2\pr{\vnabla}w_2\prt\right]+h_2\pr w_2\prt = 0 &\vr\in\partial\Omega_2/\partial\Omega_{21}
\end{matrix}
\end{align}
where  $\widehat  n_1$ and  $\widehat  n_2$ are outward normal vectors in their
respective domains. On the shared wall an energy exchange boundary condition is
applied~\citep{Billon2008}
\begin{align}
\begin{matrix}
\uvn_1{\vcdot}\left[D_1\pr{\vnabla}w_1\prt\right]+h_{11}\pr w_1\prt-h_{12}\pr w_2\prt = 0 &\vr\in\partial\Omega_{12}
\end{matrix}\\
\begin{matrix}
\uvn_2{\vcdot}\left[D_2\pr{\vnabla}w_2\prt\right]+h_{22}\pr w_2\prt-h_{21}\pr w_1\prt = 0 &\vr\in\partial\Omega_{21}
\end{matrix}
\end{align}
where the exchange coefficients $h_{11}\pr$ and $h_{22}\pr$ describe the power lost 
on their respective sides of the boundary and $h_{12}\pr$ and $h_{21}\pr$ describe power
coupled in from the other side.

If the coupling surface is a lossless reciprocal aperture with average
transmission efficiency $\tau$ then
\begin{align}
h_{12} = h_{21} = h_{22} = h_{11} = \frac{1}{4}\rc_0\tau
\end{align}
and
\begin{align}
\begin{matrix}
\uvn_1{\vcdot}\left[D_1\pr{\vnabla}w_1\prt\right]+\frac{1}{4}\rc_0\tau\left[w_1\prt-w_2\prt\right]=0 &\vr\in\partial\Omega_{12}
\end{matrix}\\
\begin{matrix}
\uvn_2{\vcdot}\left[D_2\pr{\vnabla}w_2\prt\right]+\frac{1}{4}\rc_0\tau\left[w_2\prt-w_1\prt\right]=0 &\vr\in\partial\Omega_{21}
\end{matrix}
\end{align}
In the geometric optics regime the energy lost through an aperture is indistinguishable from that
absorbed in a perfect absorber of the same area; it can therefore be argued that the energy
exchanged coefficient for an aperture should be given by
\begin{align}
\begin{matrix}
h_{ij}\pr = \rc_0\frac{\tau}{2\left(2-\tau\right)} &0\leq\tau\leq 1
\end{matrix}
\end{align}
in order to be consistent with the Jing and Xiang exchange coefficient for
absorption~(\ref{eq:ecjing}).

Denoting the net energy flux from domain $j$ into domain $i$ by 
$J_{\mathit{ij}}\prt$ these boundary conditions can also be written
\begin{align}
\begin{matrix}
\uvn_1{\vcdot}\left[D_1\pr{\vnabla}w_1\prt\right]-J_{12}\prt=0 &\vr\in\partial\Omega_{12}
\end{matrix}\\
\begin{matrix}
\uvn_2{\vcdot}\left[D_2\pr{\vnabla}w_2\prt\right]-J_{21}\prt=0 &\vr\in\partial\Omega_{21}
\end{matrix}
\end{align}
with
\begin{align}
J_{12}\prt = -J_{21}\prt = \frac{1}{4}\rc_0\tau\left[w_2\prt-w_1\prt\right].
\end{align}
Adding these and noting $\uvn_1=-\uvn_2$ we find an energy density flux continuity
boundary condition (FCBC) through the aperture~\citep{Billon2008}:
\begin{align}
\begin{matrix}
\uvn_1{\vcdot}\left[D_1\pr{\vnabla}w_1\prt-D_2\pr{\vnabla}w_2\prt\right]=0 &\vr\in\partial\Omega_{12}
\end{matrix}
\end{align}
If there is dissipative loss associated with the aperture then
\begin{align}
\uvn_1{\vcdot}\left[D_1\pr{\vnabla}w_1\prt\right]-\uvn_2{\vcdot}\left[D_2\pr{\nabla}w_2\prt\right]=\frac{1}{4}\rc_0\tau
\left[w_2\prt-w_1\prt\right]+h_{22}\pr w_2\prt-h_{11}\pr w_1\prt
\end{align}
where $h_{11},h_{22}>\frac{1}{4}\rc_0\tau$.

\section[Kantorovich reduction to two dimensions]{Kantorovich reduction to two dimensions}
\label{sc:Kant}

\subsection[Separation and reduction to 2D]{Separation and reduction to 2D}
\label{sc:Kant:red}

Systematic reduction from 3D to 2D can be made via the Kantorovich dimensional
reduction approach for domains with a uniform cross-section in one direction,
here taken to be the $z$-direction~\citep{Sequeira2012}. This separates the energy density into
\begin{align}
w\prt = W(x,y,t)Z(z),
\end{align}
where the dimensionless function $Z(z)$ is assumed known a-priori
(anzatz), for example, here it is assumed to be quadratic. This function must
satisfy the Robin BCs applied on the  $z=0$ and  $z=L_z$ faces of the domain.
If these upper and lower faces have the same uniform ECs $h_\rw$ and the
diffusivity in the domain is also uniform then
\begin{align}
Z(z) &= 1 + \xi\left( z-\frac{z^2}{L_z} \right)
\end{align}
where
\begin{align}
\xi &= \frac{h_\rw} D.
\end{align}
Using the Kantorovich dimensional reduction approach shows that the energy
density in the other two dimensions satisfies the 2D diffusion problem~\citep{Kantorovich1964}
\begin{align}
\begin{matrix}
\frac{{\partial}W\left(x,y,t\right)}{{\partial}t}
\int_0^{L_z}Z(z)^2\dif{z}
- D^\prime{\vnabla}_{xy}^2W\left(x,y,t\right)
+ \left(\Lambda_A^\prime+\Lambda_Z^\prime\right)W\left(x,y,t\right)
= p^\prime\left(x,y,t\right)
& \vr\in\Omega\\
D^\prime\uvn{\vcdot}{\vnabla}_{xy}W\left(x,y,t\right)+h^\prime\left(x,y\right)W\left(x,y,t\right)=0
&\vr\in\partial\Omega
\end{matrix}
\end{align}
where $\Omega$ is the cross-sectional area of the domain projected in the
$z$-direction, $\partial\Omega$ is its perimeter and
\begin{align}
D^\prime &= D \int_0^{L_z} Z(z)^2\dif{z} \\
\Lambda_Z^\prime &= -D \int_0^{L_z} \left(\frac{\rd^2Z\left(z\right)}{\rd z^2}Z(z)\right)\dif{z} \\
\Lambda_V^\prime &= \Lambda_V \int_0^{L_z} Z(z)^2\dif{z} \\
h^\prime\left(x,y\right) &= \int_0^{L_z} h\pr Z(z)^2\dif{z}.
\end{align}
If the $z=0$ and $z=L_z$ faces have the same uniform ECs then
\begin{align}
\int_0^{L_z} Z(z)\dif{z} &= L_z+\xi\frac{L_z^2}{6} \\
\int_0^{L_z} Z\left(z\right)^2\dif{z} &= L_z+\xi\frac{L_z^2}{3}+\xi^2\frac{L_z^3}{30} \\
\int_0^{L_z} \left(Z\left(z\right)\frac{\rd^2Z\left(z\right)}{\rd z^2}\right)\dif{z} &=-\xi\left(2+\xi \frac{L_z} 3\right).
\end{align}

\subsection[Source in two dimensions]{Sources in two dimensions}
\label{sc:Kant:src}

Note that the spurious direct energy density in 2D is determined by the 2D 
time independent Green's function so the reverberant energy density in 2D is
\begin{align}
w_\rr\pr = w\pr + \frac{P^\rt}{2\pi D} \log \left|\vr-\vr_\rs\right|.
\end{align}

\section[Finite element solution]{Finite element solution}
\label{sc:fem}

The diffusion equation can be solved numerical using a number of methods
including the finite element method (FEM). In this section we outline the FEM 
solution and its implementation using FreeFEM++~\citep{Hecht2013,Hecht2017}.

\subsection[Weak form of the diffusion problem]{Weak form of the diffusion problem}
\label{sc:fem:weak}

To derive the weak formulation of the EDM we start from the PDE
\begin{align}
\begin{matrix}
\frac{{\partial}w\prt}{{\partial}t}-{\vnabla}{\vcdot}\left[D\pr{\vnabla}w\prt\right]+\Lambda_V w\prt=p\prt &\vr\in\Omega 
\end{matrix}
\end{align}
and multiply throughout by a test function $u\pr$ and then integrate over the volume of the domain to give 
\begin{align}
\frac{{\partial}}{{\partial}t}\iiint_{\Omega} u\pr w\prt\dif{V}
-\iiint_{\Omega} u\pr{\vnabla}{\vcdot}\left[D\pr{\vnabla}w\prt\right]\dif{V}
+\Lambda_V \iiint_{\Omega } u\pr w\prt\dif{V}
=\iiint_{\Omega} u\pr p\prt\dif{V}.
\end{align}
Now assuming that all the functions are suitably smooth we apply the divergence
theorem: 
\begin{align}
\iiint_{\Omega} {\vnabla}{\vcdot}\vF\pr\dif{V} = \oiint_{\partial\Omega}\vF\pr{\vcdot}\dif{\vS}.
\end{align}
First putting $\vF\pr=\varphi\pr\vG\pr$ we obtain
\begin{align}
\iiint_{\Omega} \varphi\pr{\vnabla}{\vcdot}\vG\pr+\vG\pr{\vcdot}{\vnabla}\varphi\pr\dif{V}
=\oiint_{\partial\Omega } \varphi\pr\vG\pr{\vcdot}\dif{\vS}
\end{align}
and then letting $\vG\pr=\gamma\pr{\vnabla}\psi\pr$ gives
\begin{align}
\iiint_{\Omega} \varphi\pr{\vnabla}{\vcdot}\left[\gamma\pr{\vnabla}\psi\pr\right]+\gamma\pr{\vnabla}\psi\pr{\vcdot}{\vnabla}\varphi\pr\dif{V}
=\oiint_{\partial\Omega }\varphi\pr\gamma\pr{\vnabla}\psi\pr{\vcdot}\dif{\vS}.
\end{align}
Now letting $\varphi\pr=u\pr$, $\gamma\pr=D\pr$ and $\psi=w\prt$ we have
\begin{align}
\iiint_{\Omega} u{\vnabla}{\vcdot}\left[D{\vnabla}w\right]\dif{V}
=\oiint_{\partial\Omega } uD{\vnabla}w{\vcdot}\dif{\vS}
-\iiint_{\Omega} D{\vnabla}w{\vcdot}{\vnabla}u\dif{V},
\end{align}
which can be used to reduce the second order derivative term in the integral form above to obtain
\begin{align}
\frac{{\partial}}{{\partial}t}\iiint_{\Omega} u\pr w\prt\dif{V}
&-\oiint_{\partial\Omega} u\pr D\pr {\vnabla}w\prt{\vcdot}\dif{\vS}
+\iiint_{\Omega} D\pr{\vnabla}w\prt{\vcdot}{\vnabla}u\pr\dif{V} \nonumber\\
&+\Lambda_V \iiint_{\Omega} u\pr w\prt\dif{V}
=\iiint_{\Omega} u\pr p\prt\dif{V}.
\end{align}
The most general form of the Robin BC
\begin{align}
D\pr\uvn{\vcdot}{\vnabla} w\prt +h\pr w\prt-J_\rs\prt=0
\end{align}
is now inserted into the surface term to give the weak form of the diffusion
problem:
\begin{align}
\frac{{\partial}}{{\partial}t}\iiint_{\Omega} u\pr w\prt\dif{V}
&+\iiint_{\Omega} D\pr{\vnabla}w\prt{\vcdot}{\vnabla}u\pr\dif{V}
+\Lambda_V \iiint_{\Omega} u\pr w\prt\dif{V} \nonumber \\
&+\oiint_{\partial\Omega} u\pr h\pr w\prt\dif{S}
-\oiint_{\partial\Omega} u\pr J_\rs\prt\dif{S}
-\iiint_{\Omega} u\pr p\prt\dif{V}=0.
\end{align}
This can be written more concisely as
\begin{align}
\langle \dot{w},u \rangle + a(w,u) - \langle J_\rs+p ,u \rangle = 0,
\end{align}
where $a(w,u)$ is the variational bilinear form
\begin{align}
a(w,u)
=\iiint_{\Omega} D\pr{\vnabla}w\prt{\vcdot}{\vnabla}u\pr\dif{V}
+\Lambda_V \iiint_{\Omega} u\pr w\prt\dif{V}
+\oiint_{\partial\Omega} u\pr h\pr w\prt\dif{S}
\end{align}
and the inner product is
\begin{align}
\langle u , v \rangle
= \iiint_{\Omega} u\pr v\pr \dif{V}.
\end{align}

The above development is sufficient to implement a numerical solution of the EDM in FreeFEM++; for 
time-independent problems the essence of this implementation is:
\begin{verbatim}
   problem EDM( w , u ) 
    = int3d(Omega)       ( D * ( dx(w) * dx(u) + dy(w) * dy(u) + dz(w) * dz(u) ) )
    + int3d(Omega)       ( w * v * LambdaV )
    + int2d(Omega,dOmega)( w * u * h )   
    - int2d(Omega,dOmega)( u * Js )
    - int3d(Omega)       ( u * p );
\end{verbatim}
If the diffusivity is inhomogeneous then $D$ must be implemented using a
suitably constructed FE space or using a macro, for example
\begin{verbatim}
   func D = D1 * ( x <= partX ) + D2 * ( x > partX )  
\end{verbatim}
Similarly for the volume source term. For point sources an interpolated 
FE space can be used.

\subsection[Discretisation and finite element spaces]{Discretisation and finite element spaces}
\label{sc:fem:discrete}

The mesh is a discrete tessellation of the domain
\begin{align}
\Omega \approx \Omega_h= \bigcup_{k=1}^{n_\rT} T_k
\end{align}
using $n_\rT$ elements $T_k$ with characteristics dimension $h$:
\begin{align}
T_h = \left\{T_k:k=1,{\dots},n_\rT\right\}.
\end{align}
The elements are typically triangles in 2D or tetrahedra in 3D. The maximum edge size 
in the mesh for an accurate approximate solution is determined by the MFP in the EDM: $h\lesssim\lambda/10$.
The number of vertices in the mesh is denoted by $n_\rv$ and the boundary of the
discrete domain is denoted by $\partial\Omega\approx\partial\Omega_h$. The finite element
space on $T_h$, is denoted by $V_h\left(T_h,X\right)$:
\begin{align}
V_h\left(T_h,X\right)=\left\{w\pr:w\pr=\sum_{k=1}^M w_k\phi_k\pr,\,w_k{\in}\mathbb{R}\right\},
\end{align}
where $X$ denotes the types of finite elements. For example, piecewise continuous elements 
are denoted by $P_1$. $M$ is the dimension of $V_h$, i.e. the total number of vertices, 
which is the number of elements times the number of matching points on each 
element. $w_k$ are termed the degrees of freedom of $w$ and $M$ is the number 
of degrees of freedom (DOF). The basis functions $\phi_k\pr$ are usually defined in terms 
of barycentric coordinates. For a point $\vr{\in}T_k$ the barycentric coordinates  
$\lambda_i^k\left(\vr\right)$ are
\begin{align}
\vr = \sum_{i=1}^{n_\rrm} \lambda_i^k \vr_i
\end{align}
with
\begin{equation*}
\sum_{i=1}^{n_\rrm} \lambda _i^k=1
\end{equation*}
where $\vr_i$ are the $n_\rrm$ vertices of $T_k$. The restriction of $\phi_i$ on $T_k$ is then $\lambda_i^k$.

Now using the Galerkin method we approximate both the energy density and test function using
the same FE space
\begin{align}
w\prt \approx w_h\prt &= \sum_{j=1}^M w_j(t)\phi_j\pr \\
u\pr \approx u_h\pr &= \sum_{i=1}^M u_i\phi_i\pr
\end{align}
and substitute these into the weak form to obtain
\begin{align}
&\sum_{i=1}^M\sum_{j=1}^M u_i \dot{w}_j \iiint_{\Omega} \phi_i\pr\phi_j\pr\dif{V}
+\sum_{i=1}^M\sum_{j=1}^M u_i w_j \iiint_{\Omega} D\pr\vnabla\phi_i\pr{\vcdot}{\vnabla}\phi_j\pr\dif{V} \nonumber \\
&+\Lambda_V\sum_{i=1}^M\sum_{j=1}^M u_i w_j \iiint_{\Omega} \phi_i\pr\phi_j\pr\dif{V} 
+\sum_{i=1}^M\sum_{j=1}^M u_i w_j \oiint_{\partial\Omega} h\pr\phi_i\pr\phi_j\pr\dif{S} \nonumber \\
&-\sum_{i=1}^M u_i \oiint_{\partial\Omega} \phi_i\pr J_\rs\prt\dif{S}
-\sum_{i=1}^M u_i\iiint_{\Omega} \phi_i\pr p\prt\dif{V}
= 0.
\end{align}
Denoting the discrete matrix operators by
\begin{align}
D_{ij} &= \iiint_{\Omega} D\pr{\vnabla}\phi_i\pr{\vcdot}{\vnabla}\phi_j\pr\dif{V} \\
M_{ij} &= \iiint_{\Omega} \phi_i\pr\phi_j\pr\dif{V} \\
H_{ij} &= \oiint_{\partial\Omega} \phi_j\pr h\pr \phi_i\pr\dif{S} \\
J_i^\rs(t) &= \oiint_{\partial\Omega} \phi_i\pr J_\rs\prt\dif{S} \\
P_i(t) &= \iiint_{\Omega} \phi_i\pr p\prt \dif{V},
\end{align}
this can be written as
\begin{align}
\sum_{i=1}^M\sum_{j=1}^M u_i M_{ij} \dot{w}_j 
&+\sum_{i=1}^M\sum_{j=1}^M u_i D_{ij} w_j 
+\Lambda_V\sum_{i=1}^M\sum_{j=1}^M u_i M_{ij} w_j 
+\sum_{i=1}^M\sum_{j=1}^M u_i H_{ij} w_j \nonumber \\
&-\sum_{i=1}^M\sum_{j=1}^M u_i J^\rs_i
-\sum_{i=1}^M u_i P_i
= 0.
\end{align}
$D_{ij}$ is the stiffness matrix, $M_{ij}$ is the mass
matrix and $\Lambda_V M_{ij}+H_{ij}$ is a dissipation matrix. This can be written in matrix form as
\begin{align}
\vec{u}^\intercal \mM \dot{\vec{w}} + \vec{u}^\intercal\mD \vec{w} +\Lambda_V \vec{u}^\intercal\mM \vec{w}+ \vec{u}^\intercal\mH \vec{w}
-\vec{u}^\intercal\vec{J}^\rs -\vec{u}^\intercal\vec{P} = 0,
\end{align}
where $\vec{u}$ and $\vec{w}$ are column vectors containing the DOFs and ${}^\intercal$ denotes the transpose. 
Since the system is linear it can be solved by testing with each basis function separately and so it is
equivalent to the linear system of ordinary differential equations (ODEs)
\begin{align}
\mM \dot{\vec{w}} + \mD \vec{w} +\Lambda_V \mM \vec{w}+ \mH \vec{w}
-\vec{J}^\rs -\vec{P} = \vzero.
\end{align}
This reduces to a linear algebraic system 
\begin{align}
\left( \mD +\Lambda_V \mM + \mH \right) \vec{w} - \left( \vec{J}^\rs + \vec{P} \right) = \vzero
\end{align}
for time-independent problems. 

Time dependent systems can be solved by applying a finite difference
scheme to the time derivative. Sampling $w\prt=w(\vr,\Delta t)=w^n\pr$
and applying a Euler backward difference approximation the time
derivative term is
\begin{align}
\langle \dot{w},u \rangle \approx
\left\langle \frac{w^n-w^{n-1}}{\Delta t},u \right\rangle
=\iiint_{\Omega} u\pr  \frac{w^n\pr-w^{n-1}\pr}{\Delta t} \dif{V}.
\end{align}
The weak variational problem
\begin{align}
\left\langle \frac{w^n-w^{n-1}}{\Delta t},u \right\rangle
+ a(w,u) - \langle J_\rs+p ,u \rangle = 0,
\end{align}
can then be solved implicitly for the solution $w^n\pr$ at time step $n$
using the known solution $w^{n-1}\pr$ at the previous time-step. This 
leads to the FreeFEM++ time-stepping implementation:
\begin{verbatim}

   real dt = meshSize * meshSize / 2 / D;
   Vh w , wold = wpwb , u;

   problem EDM( w , u ) 
    = int3d(Omega)       ( w / dt * u )
    - int3d(Omega)       ( wold / dt * u )
    + int3d(Omega)       ( D * ( dx(w) * dx(u) + dy(w) * dy(u) + dz(w) * dz(u) ) )
    + int3d(Omega)       ( w * v * LambdaV )
    + int2d(Omega,dOmega)( w * u * h )   
    - int2d(Omega,dOmega)( u * Js )
    - int3d(Omega)       ( u * p );
    
   for( real t = 0 ; t < maxTime ; t += dt ) {
     wold = w;
     EDM;
   }

\end{verbatim}
This implicit scheme is unconditionally stable and so the stability criterion 
$\Delta t \leq h^2/2D$, where $h$ is the smallest edge length in the mesh, 
that applies to explicit finite-difference approximations of
the diffusion equations does not have to be satisfied. A large time-step still
has an impact on the accuracy of the solution. It appears that a value of
$\Delta t \lesssim {\Delta t}_\mathrm{ref} / 10$, where ${\Delta t}_\mathrm{ref}=L_x^2/2D$
is the characteristic diffusion time across the cavity gives a reasonable trade-off between accuracy
and computation time for the cases studied here.

\subsection[Iterative methods for coupled domains]{Iterative methods for coupled domains}
\label{sc:fem:coup}

For the case of two domains coupled through an EEBC the weak forms in each domain are
\begin{align}
\label{eq:wf:coup1}
&\frac{{\partial}}{{\partial}t}\iiint_{\Omega_1} u_1\pr w_1\prt\dif{V}
+\iiint_{\Omega_1} D_1\pr{\vnabla}w_1\prt{\vcdot}{\vnabla}u_1\pr\dif{V}
+\Lambda_{V;1}\iiint_{\Omega_1} u_1\pr w_1\prt\dif{V} \nonumber\\
&+\oiint_{\partial\Omega_1/\partial\Omega_{12}} u_1\pr h_1\pr w_1\prt\dif{S}
-\oiint_{\partial\Omega_1/\partial\Omega_{12}} u_1\pr J_{\rs;1}\prt\dif{S}
+\oiint_{\partial\Omega_{12}} u_1\pr h_{11}\pr w_1\prt\dif{S} \nonumber \\
&-\oiint_{\partial\Omega_{12}} u_1\pr J_{12}\prt\dif{S}
-\iiint_{\Omega_1} u_1\pr p_1\prt\dif{V}=0
\end{align}
and
\begin{align}
\label{eq:wf:coup2}
&\frac{{\partial}}{{\partial}t}\iiint_{\Omega_2} u_2\pr w_2\prt\dif{V}
+\iiint_{\Omega_2} D_2\pr{\vnabla}w_2\prt{\vcdot}{\vnabla}u_2\pr\dif{V}
+\Lambda_{V;2}\iiint_{\Omega_2} u_2\pr w_2\prt\dif{V} \nonumber\\
&+\oiint_{\partial\Omega_2/\partial\Omega_{21}} u_2\pr h_2\pr w_2\prt\dif{S}
-\oiint_{\partial\Omega_2/\partial\Omega_{21}} u_2\pr J_{s;2}\prt\dif{S}
+\oiint_{\partial\Omega_{21}} u_2\pr h_{22}\pr w_2\prt\dif{S} \nonumber\\
&-\oiint_{\partial\Omega_{21}} u_2\pr J_{21}\prt\dif{S}
-\iiint_{\Omega_2} u_2\pr p_2\prt\dif{V}=0,
\end{align}
where the energy density fluxes coupled between the cavities are
\begin{align}
\label{eq:wf:coup:J12}
\begin{matrix}
J_{12}\prt = h_{12}\pr w_2\prt &\vr\in\partial\Omega_{12} 
\end{matrix}\\
\label{eq:wf:coup:J21}
\begin{matrix}
J_{21}\prt = h_{21}\pr w_1\prt &\vr\in\partial\Omega_{21}
\end{matrix}
\end{align}
The system can be solved using a Robin-Robin iterative algorithm~\citep{Discacciati2007}. We first initialise
$J_{12}\prt$ using the homogeneous power balance (PWB) solution, $w_{2;\rh}$,  in $\Omega_2$
\begin{align}
\begin{matrix}
J_{12}\prt = h_{12}\pr w_{2;\rh} &\vr\in\partial\Omega_{12}
\end{matrix}
\end{align}
and then solve the FE system~(\ref{eq:wf:coup1}) in domain $\Omega_1$ using
this value to give $w_1\prt$. $J_{21}\prt$ can then be calculated from~(\ref{eq:wf:coup:J21}),
and hence the FE system~(\ref{eq:wf:coup2}) in domain $\Omega_2$ 
can be  solved for $w_2^\prt$. This process is repeated until the solution has converged; this typically only 
requires a handful of iterations: The complete iteration algorithm is:
\begin{align}
\mathrm{Initialise} &: J_{12}\prt = h_{12}\pr w_{2;\rh} \nonumber \\
\mathrm{(\ast)\,Solve\,\,for}\,\,w_1 &: \langle w_1,u_1 \rangle + a(w_1,u_1) - \langle u_1 , J_{12} \rangle - \langle u_1 , p_1 \rangle = 0 \nonumber \\
\mathrm{Update} &: J_{21}\prt = h_{21}\pr w_1 \nonumber \\
\mathrm{Solve\,\,for}\,\,w_2 &: \langle w_2,u_2 \rangle + a(w_2,u_2) - \langle u_2 , J_{21} \rangle - \langle u_2 , p_1 \rangle = 0 \nonumber \\
\mathrm{Update} &: J_{12}\prt = h_{12}\pr w_2 \nonumber \\
\mathrm{Repeat} &: \mathrm{from}\,\,(\ast) \nonumber
\end{align}

For a FCBC
\begin{align}
\uvn_1{\vcdot}\left[D_1\pr{\vnabla}w_1\prt-D_2\pr{\vnabla}w_2\prt\right]=0
\end{align}
an iterative scheme can be also be applied. The weak forms are as in~(\ref{eq:wf:coup1}) 
and~(\ref{eq:wf:coup2}), but with the $h_{11}\pr$ and $h_{22}\pr$ terms omitted - the 
corresponding bilinear terms are denoted by $\tilde{a}(w_1,u_1)$ and $\tilde{a}(w_2,u_2)$ 
respectively. Robin BCs are enforced on each side of the boundary 
\begin{align}
\uvn_1{\vcdot}D_1\pr{\vnabla}w_1\prt+J_{21}\prt &=0 \\
\uvn_2{\vcdot}D_2\pr{\vnabla}w_2\prt-J_{21}\prt &=0
\end{align}
in terms of the net energy flux $J_{21}$ from domain 1 into domain 2. At each iteration this can be
updated using a relaxation scheme with relaxation parameter $\beta\sim 0.1$~\citep{}:
\begin{align}
\mathrm{Initialise} &: J_{12}\prt = J_{21}\prt = 0 \nonumber \\
\mathrm{(\ast)\,Solve\,\,for}\,\,w_1 &: \langle w_1,u_1 \rangle + \tilde{a}(w_1,u_1) - \langle u_1 , J_{12} \rangle - \langle u_1 , p_1 \rangle = 0 \nonumber \\
\mathrm{Solve\,\,for}\,\,w_2 &: \langle w_2,u_2 \rangle + \tilde{a}(w_2,u_2) - \langle u_2 , J_{21} \rangle - \langle u_2 , p_1 \rangle = 0 \nonumber \\
\mathrm{Update} &: J_{21}\prt = \beta J_{21}\prt + ( 1 - \beta) h_{21}\pr \left[ w_1\prt - w_2\prt \right] \nonumber \\
\mathrm{Update} &: J_{12}\prt = \beta J_{12}\prt + ( 1 - \beta) h_{12}\pr \left[ w_2\prt - w_1\prt \right] \nonumber \\
\mathrm{Repeat} &: \mathrm{from}\,\,(\ast) \nonumber
\end{align}
As described in Section~\ref{sc:sum:coup} the EEBC and FCBC are equivalent in the case of a lossless aperture. 

Continuity of both the energy density and its flux can be enforced using a Schwarz domain decomposition
method~\citep{Lions1990}. The weak forms are the same as for the FCBC method, however the net flux is updated using
\begin{align}
\mathrm{Initialise} &: J_{12}\prt = J_{21}\prt = 0 \nonumber \\
\mathrm{(\ast)\,Solve\,\,for}\,\,w_1 &: \langle w_1,u_1 \rangle + \tilde{a}(w_1,u_1) - \langle u_1 , J_{12} \rangle - \langle u_1 , p_1 \rangle = 0 \nonumber \\
\mathrm{Solve\,\,for}\,\,w_2 &: \langle w_2,u_2 \rangle + \tilde{a}(w_2,u_2) - \langle u_2 , J_{21} \rangle - \langle u_2 , p_1 \rangle = 0 \nonumber \\
\mathrm{Update} &: J_{21}\prt = J_{21}\prt + \beta h_{21}\pr \left[ w_1\prt - w_2\prt \right] \nonumber \\
\mathrm{Update} &: J_{12}\prt = -J_{21}\prt \nonumber \\
\mathrm{Repeat} &: \mathrm{from}\,\,(\ast) \nonumber
\end{align}
where the sign and magnitude of $\beta$ are chosen to give a stable scheme. This Schwarz approach
is equivalent to using a single domain method. Convergence is typically slower than for the 
EEBC and FCBC schemes above.

\subsection[Two dimensional FEM solution]{Two dimensional FEM solution}
\label{sc:fem:2d}

The weak form of the reduced 2D problem is
\begin{align}
&\int_0^{L_z}Z(z)^2\dif{z}\iint_{\Omega} U\frac{{\partial}W}{{\partial}t}\dif{x}\dif{y}
+D^\prime\iint_{\Omega}\frac{{\partial}W}{{\partial}x}\frac{{\partial}U}{{\partial}x}+\frac{{\partial}W}{{\partial}y}\frac{{\partial}U}{{\partial}y}\dif{x}\dif{y} \nonumber \\
&+\left(\Lambda_V^\prime+\Lambda_Z^\prime\right)\iint_{\Omega }UW\dif{x}\dif{y}
+\oint_{\partial\Omega} Uh^\prime\left(x,y\right)W\dif{l} \nonumber \\
&-\oint_{\partial\Omega} UJ_\rc^\prime\left(x,y\right) W\dif{l}
-\iint_{\Omega} U p_\rs^\prime\left(x,y\right)\dif{x}\dif{y}=0,
\end{align}
where
\begin{align}
p_\rs^\prime\left(x,y\right)=\int_0^{L_z}Z(z)p\left(x,y,z,t\right)\dif{z}.
\end{align}
Note that this implicitly replaces the point source by a line source with the same TRP, which 
will cause errors in the vicinity of the source; these errors will increase with increasing
loss in the cavity.

For time-independent problems the 2D FreeFEM++ implementation is 
\begin{verbatim}
   problem EDM( W , U ) 
    = int2d(Omega)       ( D * ( dx(W) * dx(U) + dy(W) * dy(U) ) )
    + int2d(Omega)       ( W * V * ( LambdaV + LambdaZ ) )
    + int1d(Omega,dOmega)( W * U * h )   
    - int1d(Omega,dOmega)( U * Jc )
    - int2d(Omega)       ( U * ps );
\end{verbatim}

\section[Canonical test cases]{Canonical test cases}
\label{sc:tcs}

The canonical test cases are based on the geometry of a
cuboid cavity occupying the volume $0 \leq x \leq L_x$, $0 \leq y \leq L_y$ and 
$0 \leq z \leq L_z$ shown in Figure~\ref{fg:tcgeom}. The walls are assumed to have 
a homogeneous absorption efficiency of $\alpha_\mathrm{wall}$ and the cavity is excited
by an isotropic source of total radiated power $P^\rt_\mathrm{src}$ located at 
$(x_\mathrm{src},y_\mathrm{src},L_z/2)$. An absorbing cylinder of radius $a_\mathrm{cyl}$ 
and height $L_z$ can be positioned in the cavity, orientated with its axis in the
$z$-direction, centred at $(x_\mathrm{cyl},y_\mathrm{cyl},L_z/2)$. The cylinder is assumed
to have a homogeneous absorption efficiency of $\alpha_\mathrm{cyl}$.

The cavity can also be partitioned into two sub-cavities leaving a hole of width
$w_\mathrm{hole}$ with the full height of the cavity located in the region 
$L_y-w_\mathrm{hole} \leq y \leq L_y$, $0 \leq z \leq L_z$ of the shared 
$x=x_\mathrm{part}$ wall. The hole was chosen to span the whole cavity so that the 
dimensional reduction technique is applicable and it was located along the edge of the 
partition for experimental convenience in the validation measurements.
The thickness of the partition is denoted $t_\mathrm{part}$.

The values of the parameters are given in Table~\ref{tb:tcparam}. The wall and cylinder
absorption efficiencies were chosen to match those of the physical cavity and
cylinder used for the validation measurements described in~\citep{Flintoft2017b}.

\begin{figure}[ht]
\begin{center}
\includegraphics[width=0.8\linewidth]{figures/geometry}
\vspace{-4mm}
\caption{\label{fg:tcgeom} Cross-section in the $xy$-plane of the test case geometry.
The small black dots represent the probe locations in the z=h plane used for the measurements.}
\end{center}
\end{figure}

\begin{table}[ht]
\begin{center}
\begin{tabular}{|c|c|c|}
\hline
\textbf{Parameter}     &\textbf{Variable}      & \textbf{Value} \\
\hline
\multicolumn{3}{|c|}{\textbf{Geometrical}} \\
\hline
$L_x$                  &\texttt{Lx}            &0.90\,m \\
$L_y$                  &\texttt{Ly}            &0.45\,m \\
$L_z$                  &\texttt{Lz}            &0.45\,m \\
$x_\mathrm{part}$      &\texttt{partX}         &0.45\,m \\
$t_\mathrm{part}$      &\texttt{partThickness} &0.005\,m \\
$w_\mathrm{hole}$      &\texttt{holeWidth}     &0.04\,m \\
$x_\mathrm{src}$       &\texttt{srcX}          &0.05\,m \\
$y_\mathrm{src}$       &\texttt{srcY}          &0.225\,m \\
$z_\mathrm{src}$       &\texttt{srcZ}          &0.225\,m \\
$a_\mathrm{src}$       &\texttt{srcRadius}     &0.02\,m \\
$x_\mathrm{cyl}$       &\texttt{cylX}          &0.675/0.700\,m \\
$y_\mathrm{cyl}$       &\texttt{cylY}          &0.225\,m \\
$z_\mathrm{cyl}$       &\texttt{cylZ}          &0.225\,m \\
$a_\mathrm{cyl}$       &\texttt{cylRadius}     &0.05\,m \\
\hline
\multicolumn{3}{|c|}{\textbf{Electromagnetic}} \\
\hline
$P^\rt_\mathrm{src}$   &\texttt{srcTRP}        &1\,W \\
$\alpha_\mathrm{wall}$ &\texttt{wallAE}        &0.0027 \\
$\alpha_\mathrm{part}$ &\texttt{partAE}        &0.0027 \\
$\alpha_\mathrm{cyl}$  &\texttt{cylAE}         &0.95 \\
\hline
\end{tabular}
\end{center}
\caption{\label{tb:tcparam} Primary parameters and variable names for the test cases.}
\end{table}

\subsection[Overall solution work-flow]{Overall solution work-flow}
\label{sc:tcs:workflow}

The overall solution of the EDM is implemented using a combination of Open Source tools: 
\begin{enumerate}
 \item \textbf{Gmsh}: For 3D solutions a parametric CAD model of the geometry is created using 
 Gmsh~\citep{Geuzaine2009}. Gmsh is then used to created a tetrahedral mesh from this CAD model, which 
 is then exported using the INRIA Medit format.
 \item \textbf{FreeFEM++}: The FEM solution is determined using FreeFEM++~\citep{Hecht2013}. For 2D
 solutions the mesh is created directly within FreeFEM++, while for 3D solutions the mesh is imported in INRIA
 Medit format. The resulting energy density and energy density flux are exported in sampled form to ASCII files. 
 \item \textbf{Octave}: Post-processing is carried out by importing the sampled energy density and energy density 
 flux into GNU Octave.
\end{enumerate}
The input for Gmsh, FreeFEM++ and Octave is all taken from the same ASCII text file \texttt{parameters.geo}, which
is in Gmsh's ``geo'' format. The file contains a series of lines, each of which assigns a value to a variable:
{\small
\begin{verbatim}
isSrc = 0;             // Whether to mesh the source as a spherical surface [0/1].
isCyl = 1;             // Whether to include the cylinder [0/1].
isPart = 1;            // Whether to include the partition [0/1].
isSabine = 1;          // Whether to use Sabine or Jing & Xiang absorption factor model [0/1].
Lx = 0.9;              // Cavity size in x-direction [m].
Ly = 0.45;             // Cavity size in y-direction [m].
Lz = 0.45;             // Cavity size in z-direction [m].
wallAE = 0.0027;       // Absorption efficiency of walls [-].
partX = 0.45;          // x-coordinate of partition [m].
partThickness = 0.005; // Partition thickness [m].
partAE = 0.0027;       // Absorption efficiency of partition [-].
holeWidth = 0.04;      // Aperture width [m].
holeTE = 1.0;          // transmission efficiency of hole [-].
cylXWithPart = 0.675;  // x-coordinate of cylinder if partition is present [m].
cylXWithoutPart = 0.7; // x-coordinate of cylinder if partition is not present [m].
cylY = 0.225;          // y-coordinate of cylinder [m].
cylRadius = 0.05;      // Cylinder radius [m].
cylAE = 0.95;          // Absorption efficiency of cylinder [-].
srcX = 0.05;           // x-coordinates of source [m].
srcY = 0.225;          // y-coordinates of source [m].
srcZ = 0.225;          // z-coordinates of source [m].
srcRadius = 0.02;      // Source radius if meshed as sphere [m].
srcTRP = 1;            // Total radiated power of source [W].
\end{verbatim}}
\noindent In order for FreeFEM++ and Octave to be able to parse the file correctly it is essential 
that spaces are left on either side of the ``\texttt{=}'' sign. The order of the parameters is 
not significant. In addition to the geometrical and electromagnetic parameters defined in Table~\ref{tb:tcparam} 
four boolean parameters \texttt{isSrc}, \texttt{isCyl}, \texttt{isPart} and \texttt{isSabine} are 
also used:
\begin{itemize}
 \item \textbf{\texttt{isSrc}}: If true, a small sphere is added to the mesh to represent the source using a surface,
 otherwise an interpolated delta function source is applied;
 \item \textbf{\texttt{isCyl}}: If true, the cylinder is included in the geometry;
 \item \textbf{\texttt{isPart}}: If true, the partition is included;
 \item \textbf{\texttt{isSabine}}: If true, the Sabine loss factor~(\ref{eq:ecsabine}) is used,
 otherwise the Jing and Xiang loss factor~(\ref{eq:ecjing}) is used.
\end{itemize}



\begin{figure}[ht]
\begin{center}
\includegraphics[width=0.8\linewidth]{figures/gmshdualmesh}
\vspace{-4mm}
\caption{\label{fg:tcdualmesh} Tetrahedral mesh of the coupled cavities test-case containing the lossy cylinder.}
\end{center}
\end{figure}

\subsection[Empty unpartitioned cavity]{Empty unpartitioned cavity}
\label{sc:tcs:unpartempty}

\begin{figure}[ht]
\begin{center}
\includegraphics[width=0.6\linewidth]{figures/domains0}
\vspace{-4mm}
\caption{\label{fg:tcdom0} Domain nomenclature for the single domain implementation of the single cavity EDM.}
\end{center}
\end{figure}

\begin{table}[ht]
\begin{center}
\begin{tabular}{|l|c|c|c|c|c|}
\hline
\textbf{Power density}               &\textbf{PWB} &\multicolumn{2}{|c|}{\textbf{EDM}} \\ \cline{3-4}
{}                                   &{}           &\textbf{2D SDM} &\textbf{3D SDM}  \\
\hline
Value at centre (dB\,W\,m$^{-2})$    &28.64        &28.63           &28.63 \\
Mean (dB\,W\,m$^{-2})$               &28.64        &28.63           &28.63 \\
Minimum (dB\,W\,m$^{-2})$            &28.64        &28.60           &28.03 \\
Maximum (dB\,W\,m$^{-2})$            &28.64        &28.71           &28.74 \\
Standard deviation (dB\,W\,m$^{-2})$ &-            &6.50            &6.17  \\
Coefficient of variation (\%)        &0            &0.61            &0.57  \\
\hline
\end{tabular}
\end{center}
\caption{\label{tb:unpartempty} Statistics of the reverberant power density in the unpartitioned empty cavity.}
\end{table}

\begin{figure}[ht]
\begin{center}
\includegraphics[width=0.6\linewidth]{figures/Unpartitioned_Empty_SDM_3D_PowerDensityProfileZ}
\vspace{-4mm}
\caption{\label{fg:unpartempty_profz} Normalised vertical profile of the power density in the 2D (Kantovich anzatz) and 3D EDM of the unpartitioned empty cavity.}
\end{center}
\end{figure}

\begin{figure}[ht]
\begin{center}
\includegraphics[width=0.6\linewidth]{figures/Unpartitioned_Empty_SDM_3D_PowerDensityProfileX}
\vspace{-4mm}
\caption{\label{fg:unpartempty_profx} Power density profile in the $x$-direction along the cavity centre of the 3D EDM of the unpartitioned empty cavity.}
\end{center}
\end{figure}

\begin{figure}[ht]
\begin{center}
\includegraphics[width=0.80\linewidth]{figures/Unpartitioned_Empty_SDM_3D_EnergyDensityUniformityMap}
\includegraphics[width=0.80\linewidth]{figures/Unpartitioned_Empty_SDM_3D_EnergyDensityAnisotropyMap}
\includegraphics[width=0.80\linewidth]{figures/Unpartitioned_Empty_SDM_3D_EnergyDensityFluxMap}
\caption{\label{fg:unpartempty_maps} Energy density relative to the homogeneous PWB model (top), anisotropy (middle)
and energy density flux (bottom) in the $z$-normal plane at the half height of the cavity for the 3D EDM of the unpartitioned empty cavity.}
\end{center}
\end{figure}

\subsection[Unpartitioned cavity with a cylinder]{Unpartitioned cavity with a cylinder}
\label{sc:tcs:unpartcyl}

\begin{table}[ht]
\begin{center}
\begin{tabular}{|l|c|c|c|c|c|}
\hline
\textbf{Power density}               &\textbf{PWB} &\multicolumn{2}{|c|}{\textbf{EDM}} \\ \cline{3-4}
{}                                   &{}           &\textbf{2D SDM} &\textbf{3D SDM}  \\
\hline
Value at centre (dB\,W\,m$^{-2})$    &14.57        &15.97           &15.92 \\
Mean (dB\,W\,m$^{-2})$               &14.57        &16.08           &16.01 \\
Minimum (dB\,W\,m$^{-2})$            &14.57        &14.10           &12.52 \\
Maximum (dB\,W\,m$^{-2})$            &14.57        &17.87           &18.28 \\
Standard deviation (dB\,W\,m$^{-2})$ &-            &9.81            &9.65  \\
Coefficient of variation (\%)        &0            &23.6            &23.1 \\
\hline
\end{tabular}
\end{center}
\caption{\label{tb:unpartcyl} Statistics of the reverberant power density in the unpartitioned cavity with the cylinder.}
\end{table}

\begin{figure}[ht]
\begin{center}
\includegraphics[width=0.6\linewidth]{figures/Unpartitioned_Cylinder_SDM_3D_PowerDensityProfileZ}
\vspace{-4mm}
\caption{\label{fg:unpartcyl_profz} Normalised vertical profile of the power density in the 2D (Kantovich anzatz) and 3D EDM of the unpartitioned cavity with the cylinder.}
\end{center}
\end{figure}

\begin{figure}[ht]
\begin{center}
\includegraphics[width=0.6\linewidth]{figures/Unpartitioned_Cylinder_SDM_3D_PowerDensityProfileX}
\vspace{-4mm}
\caption{\label{fg:unpartcyl_profx} Power density profile in the $x$-direction along the cavity centre of the 3D EDM of the unpartitioned cavity with the cylinder.}
\end{center}
\end{figure}

\begin{figure}[ht]
\begin{center}
\includegraphics[width=0.80\linewidth]{figures/Unpartitioned_Cylinder_SDM_3D_EnergyDensityUniformityMap}
\includegraphics[width=0.80\linewidth]{figures/Unpartitioned_Cylinder_SDM_3D_EnergyDensityAnisotropyMap}
\includegraphics[width=0.80\linewidth]{figures/Unpartitioned_Cylinder_SDM_3D_EnergyDensityFluxMap}
\caption{\label{fg:unpartcyl_maps} Energy density relative to the homogeneous PWB model (top), anisotropy (middle)
and energy density flux (bottom) in the $z$-normal plane at the half height of the cavity for the 3D EDM of the unpartitioned cavity with the cylinder.}
\end{center}
\end{figure}

\subsection[Empty partitioned cavity]{Empty partitioned cavity}
\label{sc:tcs:emptypart}

\begin{figure}[ht]
\begin{center}
\includegraphics[width=0.6\linewidth]{figures/domains1}
\vspace{-4mm}
\caption{\label{fg:tcdom1} Domain nomenclature for a single domain implementation of the dual cavity, using
a inhomogeneous diffusivity.}
\end{center}
\end{figure}

\begin{figure}[ht]
\begin{center}
\includegraphics[width=0.6\linewidth]{figures/domains2}
\vspace{-4mm}
\caption{\label{fg:tcdom2} Domain nomenclature for a two domain implementation of the dual cavity.}
\end{center}
\end{figure}

\begin{table}[ht]
\begin{center}
\begin{tabular}{|l|c|c|c|c|c|}
\hline
\textbf{Power density}               &\textbf{PWB} &\multicolumn{4}{|c|}{\textbf{EDM}} \\ \cline{3-6}
{}                                   &{}           &\textbf{2D SDM} &\textbf{2D DDM} &\textbf{3D SDM} &\textbf{3D DDM} \\
\hline
\multicolumn{6}{|l|}{\textbf{Sub-cavity 1}} \\
\hline
Value at centre (dB\,W\,m$^{-2})$    &27.64        &28.06           &28.36           &28.05           &28.97 \\
Mean (dB\,W\,m$^{-2})$               &27.64        &28.06           &28.26           &28.05           &28.97 \\
Minimum (dB\,W\,m$^{-2})$            &27.64        &27.89           &27.42           &26.74           &27.93 \\
Maximum (dB\,W\,m$^{-2})$            &27.64        &28.14           &28.48           &28.18           &29.07 \\
Standard deviation (dB\,W\,m$^{-2})$ &-            &7.12            &6.78            &6.61            &5.91 \\
Coefficient of variation (\%)        &0            &0.81            &0.70            &0.72            &0.50 \\
\hline
\multicolumn{6}{|l|}{\textbf{Sub-cavity 2}} \\
\hline
Value at centre (dB\,W\,m$^{-2})$    &26.81        &27.75           &27.39           &27.76           &28.30 \\
Mean (dB\,W\,m$^{-2})$               &26.81        &27.76           &27.40           &27.77           &28.30 \\
Minimum (dB\,W\,m$^{-2})$            &26.81        &27.74           &27.37           &27.75           &28.27 \\
Maximum (dB\,W\,m$^{-2})$            &26.81        &27.89           &27.54           &27.90           &28.40 \\
Standard deviation (dB\,W\,m$^{-2})$ &-            &4.82            &4.50            &4.80            &3.75 \\
Coefficient of variation (\%)        &0            &0.51            &0.51            &0.50            &0.35 \\
\hline
\end{tabular}
\end{center}
\caption{\label{tb:partempty} Statistics of the reverberant power density in the partitioned empty cavity.}
\end{table}

\begin{figure}[ht]
\begin{center}
\includegraphics[width=0.6\linewidth]{figures/Partitioned_Empty_DDM_3D_PowerDensityProfileZ}
\vspace{-4mm}
\caption{\label{fg:partemptyddmeebc_profz} Normalised vertical profile of the power density in the 2D (Kantovich anzatz) and 3D EDM of the partitioned empty cavity,
using the DDM with an EEBC.}
\end{center}
\end{figure}

\begin{figure}[ht]
\begin{center}
\includegraphics[width=0.6\linewidth]{figures/Partitioned_Empty_DDM_3D_PowerDensityProfileX}
\vspace{-4mm}
\caption{\label{fg:partemptyddmeebc_profx} Power density profile in the $x$-direction along the cavity centre of the 3D EDM of the partitioned empty cavity,
using the DDM with an EEBC.}
\end{center}
\end{figure}

\begin{figure}[ht]
\begin{center}
\includegraphics[width=0.80\linewidth]{figures/Partitioned_Empty_DDM_3D_EnergyDensityUniformityMap}
\includegraphics[width=0.80\linewidth]{figures/Partitioned_Empty_DDM_3D_EnergyDensityAnisotropyMap}
\includegraphics[width=0.80\linewidth]{figures/Partitioned_Empty_DDM_3D_EnergyDensityFluxMap}
\caption{\label{fg:partemptyddmeebc_maps} Energy density relative to the homogeneous PWB model (top), anisotropy (middle)
and energy density flux (bottom) in the $z$-normal plane at the half height of the cavity for the 3D EDM of the partitioned empty cavity,
using the DDM with an EEBC.}
\end{center}
\end{figure}

\begin{figure}[ht]
\begin{center}
\includegraphics[width=0.6\linewidth]{figures/Partitioned_Empty_SDM_3D_PowerDensityProfileZ}
\vspace{-4mm}
\caption{\label{fg:partemptysdm_profz} Normalised vertical profile of the power density in the 2D (Kantovich anzatz) and 3D EDM of the partitioned empty cavity,
using the SDM.}
\end{center}
\end{figure}

\begin{figure}[ht]
\begin{center}
\includegraphics[width=0.6\linewidth]{figures/Partitioned_Empty_SDM_3D_PowerDensityProfileX}
\vspace{-4mm}
\caption{\label{fg:artemptysdm_profx} Power density profile in the $x$-direction along the cavity centre of the 3D EDM of the partitioned empty cavity,
using the SDM.}
\end{center}
\end{figure}

\begin{figure}[ht]
\begin{center}
\includegraphics[width=0.80\linewidth]{figures/Partitioned_Empty_SDM_3D_EnergyDensityUniformityMap}
\includegraphics[width=0.80\linewidth]{figures/Partitioned_Empty_SDM_3D_EnergyDensityAnisotropyMap}
\includegraphics[width=0.80\linewidth]{figures/Partitioned_Empty_SDM_3D_EnergyDensityFluxMap}
\caption{\label{fg:artemptysdm_maps} Energy density relative to the homogeneous PWB model (top), anisotropy (middle)
and energy density flux (bottom) in the $z$-normal plane at the half height of the cavity for the 3D EDM of the partitioned empty cavity,
using the SDM.}
\end{center}
\end{figure}

\subsection[Partitioned cavity with a cylinder]{Partitioned cavity with a cylinder}
\label{sc:tcs:cylpart}

\begin{table}[ht]
\begin{center}
\begin{tabular}{|l|c|c|c|c|c|}
\hline
\textbf{Power density}               &\textbf{PWB} &\multicolumn{4}{|c|}{\textbf{EDM}} \\ \cline{3-6}
{}                                   &{}           &\textbf{2D SDM} &\textbf{2D DDM} &\textbf{3D SDM} &\textbf{3D DDM} \\
\hline
\multicolumn{6}{|l|}{\textbf{Sub-cavity 1}} \\
\hline
Value at centre (dB\,W\,m$^{-2})$    &23.08        &20.83           &24.39           &20.64           &24.40 \\
Mean (dB\,W\,m$^{-2})$               &23.08        &20.81           &24.40           &20.63           &24.39 \\
Minimum (dB\,W\,m$^{-2})$            &23.08        &18.93           &21.67           &18.74           &20.41 \\
Maximum (dB\,W\,m$^{-2})$            &23.08        &21.30           &24.71           &21.50           &24.72 \\
Standard deviation (dB\,W\,m$^{-2})$ &-            &8.86            &8.27            &8.54            &8.09  \\
Coefficient of variation (\%)        &0            &6.39            &2.44            &6.18            &2.34  \\
\hline
\multicolumn{6}{|l|}{\textbf{Sub-cavity 2}} \\
\hline
Value at centre (dB\,W\,m$^{-2})$    &13.70        &14.22           &13.57           &14.25           &13.58 \\
Mean (dB\,W\,m$^{-2})$               &13.70        &15.03           &14.38           &15.05           &14.39 \\
Minimum (dB\,W\,m$^{-2})$            &13.70        &13.64           &12.99           &13.68           &13.01 \\
Maximum (dB\,W\,m$^{-2})$            &13.70        &18.85           &18.41           &18.80           &18.41 \\
Standard deviation (dB\,W\,m$^{-2})$ &-            &8.85            &8.23            &8.83            &8.22 \\
Coefficient of variation (\%)        &0            &24.1            &24.3            &23.9            &24.2 \\
\hline
\end{tabular}
\end{center}
\caption{\label{tb:partcyl} Statistics of the reverberant power density in the partitioned cavity with the cylinder.}
\end{table}

\begin{figure}[ht]
\begin{center}
\includegraphics[width=0.6\linewidth]{figures/Partitioned_Cylinder_DDM_3D_PowerDensityProfileZ}
\vspace{-4mm}
\caption{\label{fg:partcylddmeebc_profz} Normalised vertical profile of the power density in the 2D (Kantovich anzatz) and 3D EDM of the partitioned cavity with the cylinder,
using the DDM with EEBC.}
\end{center}
\end{figure}

\begin{figure}[ht]
\begin{center}
\includegraphics[width=0.6\linewidth]{figures/Partitioned_Cylinder_DDM_3D_PowerDensityProfileX}
\vspace{-4mm}
\caption{\label{fg:partcylddmeebc_profx} Power density profile in the $x$-direction along the cavity centre of the 3D EDM of the partitioned cavity with the cylinder,
using the DDM with EEBC.}
\end{center}
\end{figure}

\begin{figure}[ht]
\begin{center}
\includegraphics[width=0.80\linewidth]{figures/Partitioned_Cylinder_DDM_3D_EnergyDensityUniformityMap}
\includegraphics[width=0.80\linewidth]{figures/Partitioned_Cylinder_DDM_3D_EnergyDensityAnisotropyMap}
\includegraphics[width=0.80\linewidth]{figures/Partitioned_Cylinder_DDM_3D_EnergyDensityFluxMap}
\caption{\label{fg:partcylddmeebc_maps} Energy density relative to the homogeneous PWB model (top), anisotropy (middle)
and energy density flux (bottom) in the $z$-normal plane at the half height of the cavity for the 3D EDM of the partitioned cavity with the cylinder,
using the DDM with EEBC.}
\end{center}
\end{figure}

\begin{figure}[ht]
\begin{center}
\includegraphics[width=0.6\linewidth]{figures/Partitioned_Cylinder_SDM_3D_PowerDensityProfileZ}
\vspace{-4mm}
\caption{\label{fg:partcylsdm_profz} Normalised vertical profile of the power density in the 2D (Kantovich anzatz) and 3D EDM of the partitioned cavity with the cylinder,
with the SDM.}
\end{center}
\end{figure}

\begin{figure}[ht]
\begin{center}
\includegraphics[width=0.6\linewidth]{figures/Partitioned_Cylinder_SDM_3D_PowerDensityProfileX}
\vspace{-4mm}
\caption{\label{fg:partcylsdm_profx} Power density profile in the $x$-direction along the cavity centre of the 3D EDM of the partitioned cavity with the cylinder,
with the SDM.}
\end{center}
\end{figure}

\begin{figure}[ht]
\begin{center}
\includegraphics[width=0.80\linewidth]{figures/Partitioned_Cylinder_SDM_3D_EnergyDensityUniformityMap}
\includegraphics[width=0.80\linewidth]{figures/Partitioned_Cylinder_SDM_3D_EnergyDensityAnisotropyMap}
\includegraphics[width=0.80\linewidth]{figures/Partitioned_Cylinder_SDM_3D_EnergyDensityFluxMap}
\caption{\label{fg:partcylsdm_maps} Energy density relative to the homogeneous PWB model (top), anisotropy (middle)
and energy density flux (bottom) in the $z$-normal plane at the half height of the cavity for the 3D EDM of the partitioned cavity with the cylinder,
with the SDM.}
\end{center}
\end{figure}

\section[Conclusions]{Conclusions}
\label{sc:conc}

{\color{red} Question: How do we determine the diffusivity in cases with complex topology such as
the test case with the cylinder? Empirically it appears the diffusivity is about one quarter of that
of the empty sub-cavity in the dual cavity case. In the single cavity case the effect is not
noticeable. The fitting method of determining $\lambda_\rf = 4V/S_\rf$ does not appear to be
appropriate for single contents which significantly perturb the cavity shape.}

\bibliographystyle{myabbrvnat}
%\bibliographystyle{abbrvnat}
\bibliography{EDM_Implementation_Notes}

\end{document}
